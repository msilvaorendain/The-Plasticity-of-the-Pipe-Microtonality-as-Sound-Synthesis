\documentclass[11pt, oneside]{report}   	
\usepackage{geometry}                		% See geometry.pdf to learn the layout options. There are lots.
\geometry{a4paper}                   		% ... or a4paper or a5paper or ... 
%\geometry{landscape}                		  		
\usepackage{graphicx}				
									
\usepackage{amssymb}


\title{The Plasticity of the Pipe \\
  \large Microtonality as Sound Synthesis \\
    }
\author{Mauricio Silva Orendain}
				
\begin{document}

\maketitle

\begin{abstract}

Single organ pipes are usually considered as part of a \textit{whole} by composers and organists, since their sound cannot be individually manipulated and in order to achieve dynamics and diversity in timbre the art of registration (combination of different sets of pipes) becomes a necessity and the main way of working of one of the most complex and successful instruments in history; the organ. 

Organ pipes are conspicuously immobile and unyielding, because once they leave the organ builder's workshop, their material properties, as well as their timbre and pitch, are fixed. Since their sound cannot be individually manipulated, they are usually considered as part of a \textit{whole} by composers and organists. Despite this, through the art of registration (combination of different sets of pipes) dynamics and diversity in timbre can be achieved making the organ one of the most complex and successful instruments in history. 

Organ pipes are conspicuously immobile and unyielding, because once they leave the organ builder's workshop, their material properties, as well as their timbre and pitch, are fixed. Recent modern developments in organ building are introducing new dynamic wind-supply systems opening up a wide sound spectrum even within a single pipe, and expanding the sonic possibilities of the organ even further. 

Single pipes become versatile individual instruments opening a complete new interface for organists and composers. Registration takes another dimension in terms of crafting sound since the 



exploiting the idea of the first acoustic synthesisers 

 opening a complete new interface for organists. 



%This system has already made the organ one of the most complex, respected and rich instruments in history with "the longest and most involved history and the largest and oldest extant repertoire of any instrument in Western music". (Organ | Definition, History, Types, and Facts | Britannica”, 2024, p. 1)

  %Through the art of registration (combination of different sets of pipes) a wide variety of changes in timbre and dynamic becomes possible 
  
  %(fight against the monotony of)

%“more than any other instrumentalist, the organist feels limited in his influence on the immediate creation and dynamic of a sound.” (Ansgar Wallenhorst, p. 3)

%“the way a cello player can modulate the sound of a single note is unattainable for an organist.” (Ansgar Wallenhorst, p. 3)


Dynamic wind opens up a wide sound spectrum even within a single pipe, making it a versatile instrument and opening a complete new interface for organists. 

This artistic and scientific research project uses the term "plasticity," derived from the Greek "plastikos" (moldable, modifiable), as a metaphor and explores the moldability of the sound of individual organ pipes through the use of diverse dynamic wind supplies. 

The resulting rich palette of overtones, multiphonics, glissandi, noises, and other abstract sounds allows for a profound microscopic exploration of \textit{Klangfarbe} and a new understanding of what timbre can mean for composition.

%“Although it is one of the most complex of all musical instruments, the organ has the longest and most involved history and the largest and oldest extant repertoire of any instrument in Western music.” ([“Organ | Definition, History, Types, & Facts | Britannica”, 2024, p. 1]

Breaking stereotypes (church music) 

Stravinsky quote: the monster that never breathes 

“My first sound-image was of an all-male chorus and orchestre d'harmonie. I thought, for a moment, of the organ, but I dislike the organ's legato sostenuto and its mess of octaves, as well as the fact that the monster never breathes. The breathing of wind instruments is one of their primary attractions for me.” ([Igor Stravinsky and Robert Craft, 1963, p. 79])

%check interviews from Julliard Journal to reflect on the advantages and disadvantages of this aspect of the organ and everything around it and how your project opens up this other door that expands the possibilities and diversity of the organ.



This echoes how organists manipulate timbre through registration techniques, but on a more nuanced level resulting in a microtonal registration which can be considered a form of additive sound synthesis. 

The question of how this concept affects the architecture, sound concept and aesthetics of the organ as a \textit{whole} and the pipe as a \textit{sole} instrument in the field of organ building brings an interesting perspective. The investigation includes organs such as the Experimentalorgel inaugurated in Kassel in 2021 which alongside many other new organs and researchers on the topic serve as evidence that the exploration of organ building is not yet complete and that there is still much to explore and innovate. To achieve congruent results, such exploration requires artistic-scientific work and practice. As a consequence, the adaptation process of compositions to various organs in Europe is considered an important epistemic object of this research. This allows for a better understanding of what is transferable and what is only possible on specific organs. It is investigated how these adaptations are implemented and what aesthetic and structure decisions can be influenced by the physicality of the organ pipe. This understanding aims to build a bridge between organs for composers and organists in the field of new organ music.
\end{abstract}

\tableofcontents

\chapter{Introduction}

Going back to the conceptual origins of the organ in the Ancient Greece in the 3rd century when Ctesibus first invented the Hydraulis, we can observe that a key characteristic of this invention was the fact of not blowing the trumpet with human lungs but to actually implement a \textit{tool} to do it. This

%machine: an apparatus using mechanical power and having several parts, each with a definite function and together performing a particular task.

  when the, as its greek etymology \textit{organon} (tool in Greek) already describes it, is the concept of an organ is the implementation of a tool to create sound 

Very complex matter: clock and organ pipes as complex fields before industrial revolution 




\section{Problem}

It is known to composers and specially organists, that each organ is unique in its size, type of registers (different types of pipes) and their distribution and couplers, which is the reason organists arrive many days in advance to get to know the organ and adapt their interpretation to the possibilities on hand. Nowadays, new organs are becoming even more sophisticated and unique thanks to new technologies implemented, specially in the realm of wind-supply. In consequence, a new degree of difficulty is brought when it comes to composing and performing music using dynamic wind techniques, since most organs differ on their systems and controls (interface). Moreover, there is not much, if not none, accessible and understandable information about the behaviour of organ pipes with dynamic wind for composition and performance. Most of the time the use of dynamic wind is used either just as an effect or as an approximation to what the composer might really want. 

I had the opportunity to come in contact with a new experimental organ in Kassel which takes the wind control of the organ to a next level. This experience gave me a close insight into what is possible to achieve with such flexibility in the wind in terms of sound. Each


\section{Artistic Concept}

By treating organ pipes as individual instruments rather than as part of a larger whole, we acknowledge their versatility when interacting with dynamic wind compared to the standard static air pressure, resulting in one note producing one single timbre. This concept facilitates a deeper exploration of Klangfarbe, echoing how organists manipulate timbre through registration techniques—combining various sets of pipes to achieve specific timbres—but on a more nuanced level. We delve into a microscopic exploration of sound, incorporating overtones, multiphonics, noises, microtonality, and other abstract sounds to synthesize complex Klangfarbe. The aim is to develop a new understanding of what Klangfarbe can potentially be and its significance in composition, particularly in relation to microtonal music theory. This exploration extends to the impact on the overall sound design of organ pipes and the architecture of the organ itself, suggesting potential trends for future research, innovation among organ builders and exposing substantial information to create bridges between different organs. 

\section{Criteria}
\begin{itemize}
\item music practice oriented: knowledge gained through composing and performing 
\item part of the data should be aimed towards organ building practice to open new doors of innovations
\item create bridges between organs: outcome should facilitate organists and composers the use of dynamic wind across different organs
\item organ pipe as single instrument: focus on the use of the broad sound palette of the organ pipe under dynamic wind as single instrument and the organ as a whole organism to synthesise Klangfarbe and achieve new insights into what Klangfarbe can mean in terms of composition
\end{itemize}

\section{Music Practice}

\subsection{Goals}

\begin{itemize}
\item explore similitudes and differences across diverse organs with dynamic wind and organs without but with mechanical tracker action to explore partial registration as alternative technique
\item create real situations and problems concerning performing and composing across different organs using dynamic wind and find its possible solutions and compromises
\item treat the organ pipe as single instrument with dynamic wind to elaborate complex Klangfarbe and delve into the potential meanings of Klangbarbe in composition
\end{itemize}

\subsection{Procedure}

\subsubsection{Composing} 
composing music for dynamic wind organs and organs with mechanical tracker action 
- adapt/translate (writing individual scores for each organ) and perform the compositions in 5 different dynamic wind organs in different countries: \\
\\
- St. Martin's Church in Kassel Germany
- Musiikkitalo Organ in Helsinki, Finnland
- Dynamic Organ in Biel, Switzerland
- The Explorer in Toulouse, France\\
\\
- the 2 compositions: 
\\
Suite "Orangerie" 
Originally composed for the Große Orgel in Kassel, first two movements use only partial registration.\\
\\
"Ilmasta Puu", the whole piece needs dynamic wind.\\
Originally composed for the new Organ at the Musiikkitalo in Helsinki\\

\subsubsection{Performing} 

\section{Protocoled Data Gathering}

- run protocol with the organs where the adaptation of the pieces is being done
- run protocol in 3 - 5 other relevant organs
- run "special" protocol on Experimentalorgel in Kassel as main epistemic object of the research 

\subsection{Protocol} 

Audio recording, video recording (hand movement), score notation, spectral analysis, temperature measurement, humidity measurement and air- pressure measurement where possible  of 10 registers: four notes per octave; C, C sharp, F and F sharp: using partial registration and/vs. using the dynamic wind-supply system available.
Each pipe (note) needs a video/audio between 1.5 and 2.5min. \\
\\
Total data time per organ: 10 x 4 x1.5 = 60min or 10 x 4 x 2.5 = 100min. = around 80min. Of audiovisual content per organ. \\
\\
8 Organs = 640min. (10.5 hours) (Audio: 10GB if WAV, 48Hz and 24-bit) (Video: (?)\\
\\
\subsection{Individual Protocol Experimentalorgel}

- segregated experiments for each wind supply: wind wheel, rotary-valve, partial registration (wind volume vs wind pressure)\\
\\
- desegregated experiments: interaction of wind supplies (wind volume and wind pressure)

\subsection{Protocol Analysis and Dissemination}

Layout of the data gathered reflecting the influence of dynamic with over resulting overtones, multiphonics, glissade, noises and further  abstract noises. 

\subsubsection{Analysis}

- analyse and reflect on the similitudes, differences (pipe types and wind supply system, and their acoustic phenomena) and the necessary compromises: what aesthetic and structural decisions were taken? Where are the bridges and where are the gaps? How can the gaps be artistically solved?\\
\\






\chapter{Research Plan}

The methodology approaches the problem from three distinct perspectives in order to achieve useful and substantial results. However, due to the immensity and complexity of the topic which deals with one of the most complex and oldest instruments in history, it important to mention the awarness that there will still be a lot to be explored and experimented with. The sound scenario is so vast that there is no chance nor necessity for a final goal ...


 
Perspective one: composition and translation as investigation and documentation tool 


Perspective two: protocolled documented comparison of most common pipes in different organs and their dynamic  wind supples 


Perspective three: protocolled documentation of the sounds of the Experimentalorgel in Kassel as main epistemic object of the project and as platform for future trends in research and innovations in organ building. 


Phase 1. 

%(Check time-line click up and update) 

\cite{braaschAcousticalMeasurementsExpression2008} 

\begin{thebibliography}
\bibliographystyle{plain}
\bibliography{My Library}
\end{thebibliography}


\end{document}  

NOTES:

What research in the MA can help me base the proposal for a PhD? What data/information can help me build my PhD Exposé? --- What do I want to propose for PhD? Where? What is the final goal and fiel of research? 

1. Problem/Research Question

How can we create bridges across different organs for composers and organists using dynamic wind?

 How can we create bridges across different dynamic wind organs for composers and organists, and how does the exploration of the vast sound-palette of the organ pipe with dynamic wind reveals a new understanding of what Klangfarbe can mean for composition?

How can composers and organists present their musical works across different dynamic wind organs in Europe? 

2. Artistic Practice (to solve/answer question) 

3. Target groups

Organists, composers (organ builders) (?)

4. Procedure of the practice (methodology(?))

Composing and translation of pieces using dynamic wind. 

5. Gathering/Documenting of the practice

6. Analysis of the results

7. Conclusion and reflections

8. Dissemination




