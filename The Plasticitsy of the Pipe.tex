\documentclass[11pt, oneside]{report}   	
\usepackage{geometry}                		% See geometry.pdf to learn the layout options. There are lots.
\geometry{a4paper}                   		% ... or a4paper or a5paper or ... 
%\geometry{landscape}                		  		
\usepackage{graphicx}				
									
\usepackage{amssymb}


\title{The Plasticity of the Pipe \\
  \large Microtonality as Sound Synthesis \\
    }
\author{Mauricio Silva Orendain}
				
\begin{document}

\maketitle

\begin{abstract}
Single organ pipes are usually considered as part of a \textit{whole} by composers and organists. They themselves are conspicuously immobile and unyielding, since once they leave the organ builder's workshop, their material properties, as well as their timbre and pitch, are fixed. Even though, through the art of registration, a wide variety of changes in timbre and dynamic are possible and this system has already made the organ one of the most complex, respected and rich in history with "the longest and most involved history and the largest and oldest extant repertoire of any instrument in Western music". (Organ | Definition, History, Types, and Facts | Britannica”, 2024, p. 1)

Dynamic wind opens up a wide sound spectrum even in a single pipe, making it a versatile instrument. This artistic and scientific research project uses the term "plasticity," derived from the Greek "plastikos" (moldable, modifiable), as a metaphor and explores the moldability of the sound of individual organ pipes through the use of diverse wind supplies. The resulting rich palette of overtones, multiphonics, glissandi, noises, and other abstract sounds allows for a profound microscopic exploration of \textit{Klangfarbe} and a new understanding of what timbre can mean for composition.

%“Although it is one of the most complex of all musical instruments, the organ has the longest and most involved history and the largest and oldest extant repertoire of any instrument in Western music.” ([“Organ | Definition, History, Types, & Facts | Britannica”, 2024, p. 1]

Breaking stereotypes (church music) 

Stravinsky quote: the monster that never breathes 

“My first sound-image was of an all-male chorus and orchestre d'harmonie. I thought, for a moment, of the organ, but I dislike the organ's legato sostenuto and its mess of octaves, as well as the fact that the monster never breathes. The breathing of wind instruments is one of their primary attractions for me.” ([Igor Stravinsky and Robert Craft, 1963, p. 79])

%check interviews from Julliard Journal to reflect on the advantages and disadvantages of this aspect of the organ and everything around it and how your project opens up this other door that expands the possibilities and diversity of the organ.



This echoes how organists manipulate timbre through registration techniques, but on a more nuanced level resulting in a microtonal registration which can be considered a form of additive sound synthesis. The question of how this concept affects the architecture, sound concept and aesthetics of the organ as a \textit{whole} and the pipe as a \textit{sole} instrument in the field of organ building brings an interesting perspective. The investigation includes organs such as the Experimentalorgel inaugurated in Kassel in 2021 which alongside many other new organs and researchers on the topic serve as evidence that the exploration of organ building is not yet complete and that there is still much to explore and innovate. To achieve congruent results, such exploration requires artistic-scientific work and practice. As a consequence, the adaptation process of compositions to various organs in Europe is considered an important epistemic object of this research. This allows for a better understanding of what is transferable and what is only possible on specific organs. It is investigated how these adaptations are implemented and what aesthetic and structure decisions can be influenced by the physicality of the organ pipe. This understanding aims to build a bridge between organs for composers and organists in the field of new organ music.
\end{abstract}

\tableofcontents

\chapter{Introduction}

Going back to the conceptual origins of the organ in the Ancient Greece in the 3rd century when Ctesibus first invented the Hydraulis, we can observe that a key characteristic of this invention was the fact of not blowing the trumpet with human lungs but to actually implement a \textit{tool} to do it. This

%machine: an apparatus using mechanical power and having several parts, each with a definite function and together performing a particular task.

  when the, as its greek etymology \textit{organon} (tool in Greek) already describes it, is the concept of an organ is the implementation of a tool to create sound 

Very complex matter: clock and organ pipes as complex fields before industrial revolution 



\documentclass[11pt, oneside]{article}   	% use "amsart" instead of "article" for AMSLaTeX format
\usepackage{geometry}                		% See geometry.pdf to learn the layout options. There are lots.
\geometry{letterpaper}                   		% ... or a4paper or a5paper or ... 
%\geometry{landscape}                		% Activate for rotated page geometry
%\usepackage[parfill]{parskip}    		% Activate to begin paragraphs with an empty line rather than an indent
\usepackage{graphicx}				% Use pdf, png, jpg, or eps§ with pdflatex; use eps in DVI mode
								% TeX will automatically convert eps --> pdf in pdflatex		
\usepackage{amssymb}

%SetFonts

%SetFonts


\title{Methodology}
\author{Mauricio Silva Orendain}
%\date{}							% Activate to display a given date or no date

\begin{document}
\maketitle
\section{Criteria}

- music practice oriented (performing/composing, building)\\
\\
- outcome should be beneficial for others (organists/composers/organ builders) and create bridges between organs\\
\\
- the focus should rely on the artistic use of the broad sound palette of the organ pipe as single instrument and the organ as a whole synthesisers of sound and thus give us new insights into what Klangfarbe can mean in terms of composition \\
\section{Goals of the music practice}

- translate musical ideas using dynamic wind across organs to create real situations/problems concerning performing and composing across different organs and find its possible solutions\\
\\
- experiment through compositions the potential of partial registration for more complex Klangbarbe\\
\\
- 
\section{Music Practice}

- composing around 40min. music using dynamic wind\\
\\
- adapting and documenting the process of the music to at least 4 different organs (writing detailed score for each adaptation)\\
\\
- analyse and reflect on the adaptation: where are the bridges and where are the gaps? How can the gaps be artistically solved?

%\subsection{}




\end{document}  


1. Problem/Research Question

2. Artistic Practice (to solve/answer question) 

3. Target groups

4. Procedure of the practice (methodology(?))

5. Gathering/Documenting of the practice

6. Analysis of the results

7. Conclusion and reflections

8. Dissemination







\chapter{Research Plan}

Hola, soy el research plan 


\end{document}  

NOTES:

What research in the MA can help me base the proposal for a PhD? What data/information can help me build my PhD Exposé? --- What do I want to propose for PhD? Where? What is the final goal and fiel of research? 

1. Problem/Research Question

How can we create bridges across different organs for composers and organists using dynamic wind?

 How can we create bridges across different dynamic wind organs for composers and organists, and how does the exploration of the vast sound-palette of the organ pipe with dynamic wind reveals a new understanding of what Klangfarbe can mean for composition?

How can composers and organists present their musical works across different dynamic wind organs in Europe? 

2. Artistic Practice (to solve/answer question) 

3. Target groups

Organists, composers (organ builders) (?)

4. Procedure of the practice (methodology(?))

Composing and translation of pieces using dynamic wind. 

5. Gathering/Documenting of the practice

6. Analysis of the results

7. Conclusion and reflections

8. Dissemination




