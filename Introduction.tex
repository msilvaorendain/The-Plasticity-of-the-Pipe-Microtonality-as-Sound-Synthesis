\chapter{Introduction}

Going back to the conceptual origins of the organ in the Ancient Greece in the 3rd century when Ctesibus first invented the Hydraulis, we can observe that a key characteristic of this invention was the fact of not blowing the trumpet with human lungs but to actually implement a \textit{tool} to do it. This

%no clearly definable size or shape; during its long history the name designated many kind of instruments from hydraulic organs with a dozen pipes to a concert organ with seven manuals and thousands of pipes which can be regarded as a world record of its kind.

%machine: an apparatus using mechanical power and having several parts, each with a definite function and together performing a particular task.

  when the, as its greek etymology \textit{organon} (tool in Greek) already describes it, is the concept of an organ is the implementation of a tool to create sound 

Very complex matter: clock and organ pipes as complex fields before industrial revolution 



Parameters like phrasing and counterpoint have been wisely used by composers such as Bach in order to fight against "the monster that never breathes" \cite{igorstravinskyandrobertcraftDialoguesDiary1963} making as referred by Mozart "the queen of Instruments" one of the most complex, respected and rich in history with "the longest and most involved history and the largest and oldest extant repertoire of any instrument in Western music". (Organ | Definition, History, Types, and Facts | Britannica”, 2024, p. 1


However, like any other instrument, the big machine has its own compromises and limitations

The king of the instruments 

Organs play the music of the time, instruments influence composers and vice versa 