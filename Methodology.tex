\documentclass[11pt, oneside]{article}   	% use "amsart" instead of "article" for AMSLaTeX format
\usepackage{geometry}                		% See geometry.pdf to learn the layout options. There are lots.
\geometry{letterpaper}                   		% ... or a4paper or a5paper or ... 
%\geometry{landscape}                		% Activate for rotated page geometry
%\usepackage[parfill]{parskip}    		% Activate to begin paragraphs with an empty line rather than an indent
\usepackage{graphicx}				% Use pdf, png, jpg, or eps§ with pdflatex; use eps in DVI mode
								% TeX will automatically convert eps --> pdf in pdflatex		
\usepackage{amssymb}

%SetFonts

%SetFonts


\title{Methodology}
\author{Mauricio Silva Orendain}
%\date{}							% Activate to display a given date or no date

\begin{document}
\maketitle

\section{Criteria}

- music practice oriented (performing/composing, building)\\
\\
- outcome should be beneficial for others (organists/composers/organ builders) and create bridges between organs\\
\\
- the focus should rely on the artistic use of the broad sound palette of the organ pipe as single instrument and the organ as a whole synthesisers of sound and thus give us new insights into what Klangfarbe can mean in terms of composition \\
\section{Goals of the music practice}

- translate musical ideas using dynamic wind across organs to create real situations/problems concerning performing and composing across different organs and find its possible solutions\\
\\
- experiment through compositions the potential of partial registration for more complex Klangbarbe\\
\\
- 
\section{Music Practice}

- composing around 40min. music using dynamic wind\\
\\
- adapting and documenting the process of the music to at least 4 different organs (writing detailed score for each adaptation)\\
\\
- analyse and reflect on the adaptation: where are the bridges and where are the gaps? How can the gaps be artistically solved?

\section{Artistic Concept}

To treat organ pipes as individual instruments rather than as part of a larger whole, due to their versatility when interacting with dynamic wind, opens up the possibility of delving into Klangfarbe at a deeper level. This mirrors how organists elaborate on timbre through the art of registration but, in this case, explores another dimension (microscopic). This exploration involves the use of overtones, multiphonics, noises, microtonality, and other abstract sounds to synthesise Klangfarbe. The aim is to develop a new understanding of what Klangfarbe can potentially be and its significance in composition, particularly in relation to microtonal music theory. This exploration extends to the impact on the overall sound design of organ pipes and the architecture of the organ itself, suggesting potential trends for future research and innovation among organ builders.
\end{document}  

What research in the MA can help me base the proposal for a PhD? What data/information can help me build my Exposé? --- What do I want to propose for PhD? Where? What is the final goal and fiel of research? 

1. Problem/Research Question

How can composers and organists present their musical works across different dynamic organs in Europe? 

2. Artistic Practice (to solve/answer question) 

3. Target groups

Organists, composers (organ builders (?)

4. Procedure of the practice (methodology(?))

Composing and translation of pieces using dynamic wind. 

5. Gathering/Documenting of the practice

6. Analysis of the results

7. Conclusion and reflections

8. Dissemination




