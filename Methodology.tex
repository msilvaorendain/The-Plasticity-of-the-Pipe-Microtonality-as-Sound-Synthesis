\documentclass[11pt, oneside]{report}   	% use "amsart" instead of "article" for AMSLaTeX format
\usepackage{geometry}                		% See geometry.pdf to learn the layout options. There are lots.
\geometry{letterpaper}                   		% ... or a4paper or a5paper or ... 
%\geometry{landscape}                		% Activate for rotated page geometry
%\usepackage[parfill]{parskip}    		% Activate to begin paragraphs with an empty line rather than an indent
\usepackage{graphicx}				% Use pdf, png, jpg, or eps§ with pdflatex; use eps in DVI mode
								% TeX will automatically convert eps --> pdf in pdflatex		
\usepackage{amssymb}

%SetFonts

%SetFonts


\title{%
 The Plasticity of the Pipe \\
  \large Microtonality as Sound Synthesis \\
    }
\author{Mauricio Silva Orendain}
%\date{}							% Activate to display a given date or no date

\begin{document}
\maketitle

\chapter*{Methodology}

\section{Artistic Concept}

By treating organ pipes as individual instruments rather than as part of a larger whole, we acknowledge their versatility when interacting with dynamic wind compared to the standard static air pressure, resulting in one note producing one timbre. This concept facilitates a deeper exploration of Klangfarbe, echoing how organists manipulate timbre through registration techniques—combining various sets of pipes to achieve specific timbres—but on a more nuanced level. We delve into a microscopic exploration of sound, incorporating overtones, multiphonics, noises, microtonality, and other abstract sounds to synthesize complex Klangfarbe. The aim is to develop a new understanding of what Klangfarbe can potentially be and its significance in composition, particularly in relation to microtonal music theory. This exploration extends to the impact on the overall sound design of organ pipes and the architecture of the organ itself, suggesting potential trends for future research and innovation among organ builders.

\section{Criteria}

- music practice oriented (performing/composing, building)\\
\\
- outcome should be beneficial for others (organists/composers/organ builders) and create bridges between organs\\
\\
- the focus should rely on the artistic use of the broad sound palette of the organ pipe as single instrument and the organ as a whole synthesiser of sound and thus give us new insights into what Klangfarbe can mean in terms of composition \\

\section{Goals of Music Practice}

- explore similitudes and differences across diverse dynamic wind organs including as well organs without dynamic wind systems but with mechanical tracker action to explore partial registration as a dynamic wind alternative technique\\
\\
- create real situations/problems concerning performing and composing across different organs using dynamic wind and find its possible solutions and compromises\\
\\
- portrait the potential meaning of Klangbarbe in compositions\\
\\
- 
\section{Music Practice}

\subsection{Composing} 
composing music for dynamic wind organs and organs with mechanical tracker action 
- adapt/translate (writing individual scores for each organ) and perform the compositions in 5 different dynamic wind organs in different countries: \\
\\
- St. Martin's Church in Kassel Germany
- Musiikkitalo Organ in Helsinki, Finnland
- Dynamic Organ in Biel, Switzerland
- The Explorer in Toulouse, France\\
\\
- the 2 compositions: 
\\
Suite "Orangerie" 
Originally composed for the Große Orgel in Kassel, first two movements use only partial registration.\\
\\
"Ilmasta Puu", the whole piece needs dynamic wind.\\
Originally composed for the new Organ at the Musiikkitalo in Helsinki\\
\\
\subsection{Analysis}
- analyse and reflect on the similitudes, differences (pipe types and wind supply system, and their acoustic phenomena) and the necessary compromises: what aesthetic and structural decisions were taken? Where are the bridges and where are the gaps? How can the gaps be artistically solved?\\
\\
\section{Protocoled Data Gathering}

- run protocol with the organs where the adaptation of the pieces is being done
- run protocol in 3 - 5 other relevant organs
- run "special" protocol on Experimentalorgel in Kassel as main epistemic object of the research 

\subsection{Protocol} 

Audio recording, video recording (hand movement), score notation, spectral analysis, temperature measurement, humidity measurement and air- pressure measurement where possible  of 10 registers: four notes per octave; C, C sharp, F and F sharp: using partial registration and/vs. using the dynamic wind-supply system available.
Each pipe (note) needs a video/audio between 1.5 and 2.5min. \\
\\
Total data time per organ: 10 x 4 x1.5 = 60min or 10 x 4 x 2.5 = 100min. = around 80min. Of audiovisual content per organ. \\
\\
8 Organs = 640min. (10.5 hours) (Audio: 10GB if WAV, 48Hz and 24-bit) (Video: (?)\\
\\
\subsection{Individual Protocol Experimentalorgel}

- segregated experiments for each wind supply: wind wheel, rotary-valve, partial registration (wind volume vs wind pressure)\\
\\
- desegregated experiments: interaction of wind supplies (wind volume and wind pressure)

\subsection{Protocol Analysis and Dissemination}

Layout of the data gathered reflecting the influence of dynamic with over resulting overtones, multiphonics, glissade, noises and further  abstract noises. 

\end{document}  

What research in the MA can help me base the proposal for a PhD? What data/information can help me build my Exposé? --- What do I want to propose for PhD? Where? What is the final goal and fiel of research? 

1. Problem/Research Question

How can we create bridges across different dynamic wind organs for composers and organists?

 How can we create bridges across different dynamic wind organs for composers and organists, and how does the exploration of the vast sound-palette of the organ pipe with dynamic wind reveals a new understanding of 

How can composers and organists present their musical works across different dynamic wind organs in Europe? 

2. Artistic Practice (to solve/answer question) 

3. Target groups

Organists, composers (organ builders (?)

4. Procedure of the practice (methodology(?))

Composing and translation of pieces using dynamic wind. 

5. Gathering/Documenting of the practice

6. Analysis of the results

7. Conclusion and reflections

8. Dissemination




