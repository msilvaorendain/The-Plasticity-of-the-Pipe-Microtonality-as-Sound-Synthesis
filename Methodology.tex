\chapter{Methodology}

\section{Problem}

It is known to composers and specially organists, that each organ is unique in its size, type of registers (different types of pipes) and their distribution and couplers, which is the reason organists arrive many days in advance to get to know the organ and adapt their interpretation to the possibilities on hand. Nowadays, new organs are becoming even more sophisticated and unique thanks to new technologies implemented, specially in the realm of wind-supply. In consequence, a new degree of difficulty is brought when it comes to composing and performing music using dynamic wind techniques, since most organs differ on their systems and interfaces. Moreover, there is not much, if not none, accessible and understandable information about the behaviour of single organ pipes with dynamic wind for composition and performance. One of the main reasons behind is because it is something rather new in the long history of organ. Most of the time the use of dynamic wind is used either just as an effect or as an approximation to what the composer might really want without delving into more specific composing because of the fear of not being able to perform it the same way in other organs. This however, is anyways already a problem with normal repertoire since organists end up having to co-compose important parameters of the music when they do the registration in different organs. This might be connected to what Stravinsky meant when he mentioned "its mess of octaves" (Igor Stravinsky and Robert Craft, 1963, p. 79) while arguing why he did not use the organ for his Symphony of Psalms. 

I had the opportunity to come in contact with a new experimental organ in Kassel which takes the wind control of the organ and of its single pipes to a next level. This experience gave me a close insight into what is possible to achieve with such flexibility in the wind in terms of sound design and Klangfarbe. Each


\section{Artistic Concept}

By treating organ pipes as individual instruments rather than as part of a larger \textit{whole}, we acknowledge their versatility when interacting with dynamic wind compared to the standard static air pressure, resulting in one note producing one single timbre. This concept facilitates a deeper exploration of Klangfarbe, echoing how organists manipulate timbre through registration techniques—combining various sets of pipes to achieve specific timbres—but on a more nuanced level. We delve into a microscopic exploration of sound, incorporating overtones, multiphonics, noises, microtonality, and other abstract sounds to synthesize complex Klangfarbe. The aim is to develop a new understanding of what Klangfarbe can potentially be and its significance in composition, particularly in relation to microtonal music theory. This exploration extends to the impact on the overall sound design of organ pipes and the architecture of the organ itself, suggesting potential trends for future research, innovation among organ builders and exposing substantial information to create bridges between different organs. 

\section{Criteria}

\begin{itemize}
\item music practice oriented: knowledge gained through composing and performing 
\item part of the data should be addressed towards organ building practice to open new venues for innovations
\item create bridges between organs: outcome should facilitate organists and composers the use of dynamic wind across different organs
\item organ pipe as single instrument: explore its moldability through dynamic wind and the resulting broad sound palette as individual expressive instrument 
\item organ as a \textit{whole}:  delve into its etymologic meaning \textit{organon} and treat is as a tool (an organism) to synthesise (put together) Klangfarbe 
\item achieve new insights into what Klangfarbe can mean in terms of composition
\end{itemize}

\section{Music Practice}

\subsection{Goals}

\begin{itemize}
\item explore similitudes and differences across diverse organs with dynamic wind and organs without but with mechanical tracker action to explore partial registration as alternative technique
\item create real situations and problems concerning performing and composing across different organs using dynamic wind and find its possible solutions and compromises
\item treat the organ pipe as single instrument with dynamic wind to elaborate complex Klangfarbe and delve into the potential meanings of Klangbarbe in composition
\end{itemize}

\subsection{Procedure}

\subsubsection{Composing} 
composing music for dynamic wind organs and organs with mechanical tracker action 
- adapt/translate (writing individual scores for each organ) and perform the compositions in 5 different dynamic wind organs in different countries: \\
\\
- St. Martin's Church in Kassel Germany
- Musiikkitalo Organ in Helsinki, Finnland
- Dynamic Organ in Biel, Switzerland
- The Explorer in Toulouse, France\\
\\
- the 2 compositions: 
\\
Suite "Orangerie" 
Originally composed for the Große Orgel in Kassel, first two movements use only partial registration.\\
\\
"Ilmasta Puu", the whole piece needs dynamic wind.\\
Originally composed for the new Organ at the Musiikkitalo in Helsinki\\

\subsubsection{Performing} 

\section{Protocoled Data Gathering}

- run protocol with the organs where the adaptation of the pieces is being done
- run protocol in 3 - 5 other relevant organs
- run "special" protocol on Experimentalorgel in Kassel as main epistemic object of the research 

\subsection{Protocol} 

Audio recording, video recording (hand movement), score notation, spectral analysis, temperature measurement, humidity measurement and air- pressure measurement where possible  of 10 registers: four notes per octave; C, C sharp, F and F sharp: using partial registration and/vs. using the dynamic wind-supply system available.
Each pipe (note) needs a video/audio between 1.5 and 2.5min. \\
\\
Total data time per organ: 10 x 4 x1.5 = 60min or 10 x 4 x 2.5 = 100min. = around 80min. Of audiovisual content per organ. \\
\\
8 Organs = 640min. (10.5 hours) (Audio: 10GB if WAV, 48Hz and 24-bit) (Video: (?)\\
\\
\subsection{Individual Protocol Experimentalorgel}

- segregated experiments for each wind supply: wind wheel, rotary-valve, partial registration (wind volume vs wind pressure)\\
\\
- desegregated experiments: interaction of wind supplies (wind volume and wind pressure)

\subsection{Protocol Analysis and Dissemination}

Layout of the data gathered reflecting the influence of dynamic with over resulting overtones, multiphonics, glissade, noises and further  abstract noises. 

\subsubsection{Analysis}

- analyse and reflect on the similitudes, differences (pipe types and wind supply system, and their acoustic phenomena) and the necessary compromises: what aesthetic and structural decisions were taken? Where are the bridges and where are the gaps? How can the gaps be artistically solved?\\
\\



